\section{Prefetcher Configurations}
\label{Config}

This section describes the approach in putting together all the
prefetching components across the cache hierarchies.

\subsection{Single-Core Configuration}
\label{Config-Single}

\noindent \textbf{1st Level Data Cache:} For the L1D Cache we are
using a modified version of the next-N-line prefetcher~\cite{nextn}.
The basic next line prefetcher is modified to have a small table
containing the last block accessed by a page. The table is indexed by
hashing the page number of an access.  When an access occurs, the
current block access and the previous access are compared.  If the
delta between accesses is +1, then a score table is indexed by the
page number and its value increased. If the delta is not +1, it is
decreased. When prefetching, the score table is accessed and if the
value is above a specified threshold, the next cache line after the
access is prefetched. This throttling allows for the prefetcher to be
aware if the page is susceptible to multiple +1 deltas, usually
consecutively. If the page does not benefit from next line
prefetching, the prefetcher is turned off so that it does not risk
polluting the L1D and wrongfully evicting data that is more beneficial
to performance.

The prefetcher continues to prefetch the next N consecutive lines of
that page. N is obtained dynamically by sharing the Prefetch Queue
(PQ) resources over consecutively active pages, as explained in
Section~\ref{Enhancements-Misc}

All the prefetch suggestions coming from the L1D prefetcher are placed
in the L1D Cache.

\noindent \textbf{2nd Level Cache:} For the L2C, we are using the
enhanced SPP+PPF approach described in the
section~\ref{Enhancements}. Prefetches originating from the L2C can be
placed in L2C or LLC, depending on the confidence estimate given by
the perceptron sum.

\textcolor{blue}{ - Talk about ignoring previous level's prefetches or
  not
% djimenez: you don't have to discuss every detail. that's what the code is for.
}

\noindent \textbf{Last Level Cache:} The LLC prefetcher is the basic
next-line prefetcher. The prefetcher gets triggered on demand accesses
and the prefetch accesses originating from the L1 prefetcher.

\begin{table}[h]
\begin{adjustwidth}{}{}
    \centering
    \begin{adjustbox}{width=0.8\columnwidth}
    \begin{tabular}{|c|c|c|c|}
    \hline
        \textbf{Structure} &
        \textbf{Entry} &
        \textbf{Components} &
        \textbf{Total} \\
    \hline
                                            &  \multirow{5}{0.5cm}{256}    & Valid (1 bit)  &             \\
                                             &      & Tag (16 bits)        &  \multirow{2}{0.9cm}{11008}           \\
                            Signature Table  &   & Last Offset (6 bits) &  \multirow{2}{0.5cm}{bits}  \\  
                                             &      & Signature (12 bits)  &             \\
                                             &      & LRU (6 bits)         &             \\
    \hline
                                    &  \multirow{3}{0.5cm}{512}    & $C_{sig}$ (4bits)      &\multirow{2}{0.9cm}{24576}               \\
                       Pattern Table         &   & $C_{delta}$ (4*4 bits) &  \multirow{2}{0.5cm}{bits}  \\
                                             &      & Delta (4*7 bits)       &               \\
    \hline
        \multirow{4}{1.5cm}{Perceptron\newline}     & 4096*4    & \multirow{4}{0.8cm}{5 bits}  & \multirow{3}{1.1cm}{113280}             \\
        \multirow{3}{1.2cm}{Weights}                & 2048*2    &           &  \multirow{3}{0.5cm}{bits}  \\
                                                    & 1024*2    &           &               \\
                                                    & 128*1     &           &              \\
    \hline
        Prefetch                & \multirow{2}{0.7cm}{1024}      & \multirow{2}{1cm}{85 bits}       & 87040 \\
        Table\footnotemark[1]   &           &               & bits\\
    \hline
        Reject                & \multirow{2}{0.7cm}{1024}      & \multirow{2}{1cm}{84 bits}    & 86016 \\
        Table\footnotemark[2] & & & bits\\
    \hline
        \multirow{4}{1.0cm}{Global\newline\newline}   & \multirow{4}{0.2cm}{8} & Signature (12 bits)  & \multirow{4}{1.1cm}{264 bits} \\
        \multirow{3}{1.1cm}{History\newline}        &                        & Confidence (8 bits)  &                               \\
        \multirow{2}{1.2cm}{Register}               &                        & Last Offset (6 bits) &                               \\
                                                    &                        & Delta (7 bits)       &                               \\
    \hline
        Accuracy        & 1     & C$_{total}$       & 10 bits   \\
        Counters        & 1     & C$_{useful}$      & 10 bits   \\
    \hline
        \multirow{3}{1.5cm}{Global PC\newline}      &       & $PC_1$ (12 bits)      &           \\
        \multirow{2}{1.5cm}{~Trackers}              & 3     & $PC_2$ (12 bits)      & 36 bits   \\
                                                    &       & $PC_3$ (12 bits)      &           \\
    \hline
        \multicolumn{4}{|c|}{\textbf{Total: 322,240 bits = 39.34 KB}}\\
    \hline
    \end{tabular}
    \end{adjustbox}
    \caption{Prefetcher Storage Cost}
    \label{tab:PPF_overhead}
\end{adjustwidth}
\end{table}

\subsection{Single-Core Complexity}
\label{Config-Complex}

Table~\ref{tab:PPF_overhead} \textcolor{blue}{(CURRENTLY INCOMPLETE)}
shows a detailed analysis of the hardware overhead required to
implement the three prefetchers. It is well within the championship
budget of 64KB. In terms of L2C prefetcher's logical complexity, SPP
is a cascade of three tables, with the output of one indexing into the
next. Constructing the signature only requires simple operations like
shifting and XOR. PPF requires parallel indexing into nine different
tables and adding nine 5-bit integers, which is well within the
complexity of currently implemented perceptron-based branch
predictors.  Weight updates are done only in steps of +1 or -1.
\textcolor{red}{Do we want to max out the tables to use all the
  available space?  If that gives us any benefit we should, we don't
  get anything extra for being smaller than the required space right?}


\subsection{Multi-Core Configuration}
\label{Config-Multi}

When in a multi-core configuration, we only using the L2C Prefetcher
as described above. We observed that the efficient filtering mechanism
in PPF helps avoid pollution in the shared LLC, while the other level
prefetchers introduced some noise such that better performance is
observed when they are turned off.  Thus, we leverage NUM_CPUs to
disable the L1 and LLC prefetchers.  Regardless the overall overhead
matches that of the single core arrangement and thus is within the
championship rules.
