\section{Introduction}
\label{Introduction}

All prefetchers are designed around a fundamental trade-off between
two important metrics: coverage and accuracy. Prefetcher coverage
refers to the fraction of baseline cache misses that the prefetcher
pulls into the cache prior to their reference. Prefetcher
accuracy refers to the fraction of prefetched cache lines that end up
being used by the application. Our key idea in this paper is to 
efficiently overcome this trade-off. This can give us a prefetching 
mechanism which can learn complex pointer-chasing patterns (high coverage)
and yet work good on constrained multi-core systems (high accuracy).

Another challenging aspect of this championship is to have a coordinated 
prefetch design approach. This involves controlling the inter-hierarchy 
prefetch communication like prefetch misses from L1D appearing as accesses 
to L2C. Another aspect to consider is placement of the incoming prefetches.
Since now we have control over all the three cache hierarchies, a correct 
prefetch suggestion placed in the wrong level, say L1D, can potentially 
do more harm by wasting precious resources.

In this paper, we follow a modular approach of explaining our basic prefetch
block, SPP. We follow that by adding the enhancements we did to tackle the 
trade-off explained above. Finally, we talk about fitting together all the 
pieces across the cache hierarchies to get the final prefetching mechanism 
implemented.

In a single core configuration running a mix of memory intensive SPEC CPU 
2017 traces, \_\_\_ increases performance by XX\% compared to no prefetching. 
In a four-core system, \_\_\_ saw an improvement of XX\% over the baseline.
