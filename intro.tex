\section{Introduction}
\label{Introduction}

Prefetchers are designed around a fundamental trade-off between two important
metrics: coverage and accuracy. Prefetcher coverage refers to the fraction of
baseline cache misses that the prefetcher brings to the cache before their
reference. Accuracy refers to the fraction of prefetched cache lines that are
actually used by the application. The key idea in this paper is to efficiently
balance this trade-off. Our approach yields a prefetching mechanism that can
learn complex pointer-chasing patterns (high coverage) and yet work well on
constrained multi-core systems (high accuracy).

A challenging aspect of the championship is to have a coordinated 
prefetch design approach. This involves controlling the inter-hierarchy 
prefetch communication like prefetch misses from L1D appearing as accesses 
to L2C. Another aspect to consider is placement of the incoming prefetches.
Since we now have control over all the three cache hierarchies, a correct 
prefetch suggestion placed in the wrong level, say L1D, can potentially 
do more harm by wasting precious resources.

In this paper, we follow a modular approach of explaining our basic
prefetcher, SPP. We then explain the enhancements we did to address the
trade-off explained above. Finally, we discuss fitting together the pieces
across the cache hierarchies to get the final prefetching mechanism
implemented.

In a single core configuration, running a mix of memory intensive SPEC CPU 
2017 traces, \_\_\_ increases performance by XX\% compared to no prefetching. 
In a four-core system, \_\_\_ saw an improvement of XX\% over the baseline.

% djimenez: can we just call it PPF? 
