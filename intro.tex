\section{Introduction}
\label{Introduction}

Prefetchers are designed around a fundamental trade-off between two
important metrics: coverage and accuracy. Prefetcher coverage refers
to the fraction of baseline cache misses that the prefetcher brings to
the cache before their reference. Accuracy refers to the fraction of
prefetched cache lines that are actually used by the application. The
key idea in this paper is to efficiently balance this trade-off. Our
approach yields a prefetching mechanism that can learn complex
pointer-chasing patterns (high coverage) and yet work well on
constrained multi-core systems (high accuracy).

A key challenge in prefetching for multi-level cache hierarchies lies
in designing a coordinated prefetch design approach.  This involves
controlling the inter-hierarchy prefetch communication like prefetch
misses from L1D appearing as accesses to L2C.  Another aspect to
consider is placement of the incoming prefetches.  A correct prefetch
suggestion placed in the wrong level, say L1D, can potentially do more
harm by wasting precious resources.

In this paper, we follow a modular approach of explaining our basic
prefetcher -- SPP, our perceptron prefetch filtering scheme -- PPF, and the
further enhancements included to address the trade-offs described
above.  Finally, we discuss fitting together the pieces across the
cache levels in the final prefetching mechanism implemented.

In a single core configuration, running a mix of memory intensive SPEC
CPU 2017 traces, our prefetcher increases performance by 40.4\% compared to no
prefetching.  In a four-core system, it saw an improvement of XX\%
over the baseline.

% djimenez: can we just call it PPF? 
% eshan: calling it 'our prefetcher' now
